% !TeX program = xelatex
\documentclass[a4paper]{scrarticle}

\usepackage[ngerman]{babel}
\usepackage[utf8]{inputenc}
\usepackage[T1]{fontenc}

\usepackage[left=2.25cm, right=2.25cm, top=3.00cm, bottom=3.50cm]{geometry}
\usepackage[headsepline, footsepline]{scrlayer-scrpage}
\renewcommand*{\headfont}{\normalfont}
\usepackage[backend=biber,style=apa]{biblatex} 
	\addbibresource{bibliography.bib}
	\bibliography{bibliography}
\usepackage{csquotes}

\usepackage{graphicx}
\usepackage[figurename=Fig.]{caption}

\usepackage{amsmath}
\usepackage{amssymb}

\usepackage{tabto}
\usepackage{xcolor}
\usepackage{enumitem}
\usepackage{blindtext, showframe}
\usepackage{fontspec}
\usepackage{lipsum} % for dummy text

\definecolor{AKSAcolor}{rgb}{0.64,0.44,0.32}
\newfontfamily\AKAfont{AKA}

% Redefine sectioning commands to set font
\makeatletter
\renewcommand\section{\@startsection {section}{1}{\z@}%
                                   {-3.5ex \@plus -1ex \@minus -.2ex}%
                                   {2.3ex \@plus.2ex}%
                                   {\Huge\AKAfont}}
\renewcommand\subsection{\@startsection{subsection}{2}{\z@}%
                                     {-3.25ex\@plus -1ex \@minus -.2ex}%
                                     {1.5ex \@plus .2ex}%
                                     {\Large\AKAfont}}
\renewcommand{\maketitle}{%
																		 \begin{titlepage}
																			 \null\vfill % Add space at the top
																			 \begin{center}
																				 {\huge\@title\par}%
																				 \vspace{0.5cm} % Adjust spacing between title and author
																				 {\large\@subtitle\par} % Add the subtitle
																				 \vspace{1.5cm} % Adjust spacing between author and subtitle
																				 {\Large\@author\par}
																				 \vspace{0.5cm} % Adjust spacing between subtitle and date
																				 {\large\@date\par} % Add the date
																			 \end{center}
																			 \vfill % Fill remaining space at the bottom
																			 \begin{center}
																				Eingereicht bei Michael Kappeler
																			 \end{center}
																			 \@thanks % If you have any thanks or notes, they will be printed here
																		 \end{titlepage}%
																	 }

\makeatother

\usepackage{hyperref}
\hypersetup{colorlinks=false, pdfborder={0 0 0}, pdftitle=}



\begin{document}

\title{\AKAfont\Huge\textcolor{AKSAcolor}{Algorithmische Skitourenplanung}}
\subtitle{Vom Bildschirm an den Berg – und zurück}
\author{Jesse Born, G21D}
\date{September 2024}

\pagenumbering{gobble} % Suppress page numbering
\ihead{Algorithmische Skitourenplanung}
\ifoot{\\Jesse Born, 2024}

\maketitle
\clearpage


\addcontentsline{toc}{section}{Inhaltsverzeichnis}
\tableofcontents
\newpage
\begin{abstract}
	\section*{Abstract}
	\addcontentsline{toc}{section}{Abstract}
	In vorliegender Arbeit wird die methodische sowie technische Umsetzung eines Computer-Programs zur automatisierten Planung von Ski- \& Bergtouren disskutiert. 
	Auf Basis des digitalen Höhenmodels der Schweiz SwissAlti\textsuperscript{3D}, der Bodenbedeckungskarte SwissTLM\textsuperscript{3D} sowie historischen Unfalldaten wird eine computeroptimierte Reduktionsmethode entwickelt, welche flächendeckend individuelle Gefahrenwerte für einzelne Rasterpunkte innerhalb der Schweiz errechnen kann.
	So können mittels vom Benutzer eingetragenen Start- \& Zielkoordinaten sichere, bzw. risikooptimierte Routen generiert werden.
\end{abstract}
\section*{Vorwort}
\addcontentsline{toc}{section}{Vorwort}
Jedes Jahr ereignen sich in den Alpen abseits der gesicherten Pisten der Skigebiete XXX tödliche verlaufende Unfälle. Fortschritte 
\clearpage
\pagenumbering{arabic}

\section{Einleitung}
\subsection{Theoretische Grundlagen}
\lipsum[6-10]
\subsection{Methodik}
\lipsum[6-10]
\clearpage
\section{Hauptteil}
\subsection{Resultat}
\lipsum[6-10]
\subsection{Produktevaluation}
\lipsum[6-10]
\subsection{(Methodische Reflexion)}
\lipsum[6-10]
\clearpage
\section{Schlusswort, Fazit und Diskussion}
\lipsum[6-10]
\clearpage
\section{Literaturverzeichnis}
\lipsum[6-10]
\clearpage
\section{Abbildungsverzeichnis}
\lipsum[6-10]
\clearpage


\end{document}