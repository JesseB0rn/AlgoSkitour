% !TeX program = xelatex
\documentclass[a4paper, listof=numbered, numbers=noenddot]{scrarticle}

\usepackage{style}
\usepackage{pdfpages}

\includeonly
{
    sections/abstract,
    sections/vorwort,
		sections/einleitung,
		sections/hauptteil,
    sections/schlusswort,
    sections/anhang,
    sections/glossar,
}


\title{\AKAfont\Huge\textcolor{AKSAcolor}{Algorithmische\\Skitourenplanung}}
\subtitle{Vom Bildschirm an den Berg -- und zurück}
\author{Jesse Born, G21D}
\date{Oktober 2024}

% \ihead{Algorithmische Skitourenplanung}
% \ihead{\leftmark}
% \cfoot{}
% \ofoot{\pagemark}

\newacronym{llb}{LLB}{Lawinenlagebericht}
\newacronym{gis}{GIS}{Geoinformationssystem}
\newacronym{dhm}{DHM}{digitales Höhenmodell}
\newacronym{dem}{DEM}{Digital Elevation Model}
\newacronym{cli}{CLI}{Command Line Interface, Kommandozeile}
\newacronym{grm}{GRM}{Grafische Reduktionsmethode}
\newacronym{qrm}{QRM}{Quantitative Reduktionsmethode}
\newacronym{gpu}{GPU}{Grafikprozessor}
\newacronym{unix}{UNIX}{UNiplexed Information Computing System}
\newacronym{gdal}{GDAL}{Geospatial Data Abstraction Library}
\newacronym{slf}{SLF}{WSL-Institut für Schnee-und Lawinenforschung}
\newacronym{sac}{SAC}{Schweizer Alpen-Club}
\newacronym{lvs}{LVS}{Lawinenverschüttetensuchgerät}
\newacronym{bsd}{BSD}{Berkeley Software Distribution}
\newacronym{gb}{GB}{GigaByte}
\newacronym{GeoTIFF}{GeoTIFF}{Georeferenced Tagged Image File Format}
\newacronym{GPS}{GPS}{Global Positioning System}
\newacronym{lts}{LTS}{Long Term Support}
\newacronym{qgis}{QGIS}{Quantum Geographic Information System}
\newacronym{ramms}{RAMMS}{Rapid Mass Movement Simulator}
\newacronym{rgb}{RGB}{Red Green Blue}
\newacronym{url}{URL}{Uniform Resource Locator}
\newacronym{wsl}{WSL}{Eidg. Forschungsanstalt für Wald, Schnee und Landschaft}

% \newacronym{ai}{AI}{Kanton Appenzell Innerrhoden}
% \newacronym{be}{BE}{Kanton Bern}
% \newacronym{gl}{GL}{Kanton Glarus}
% \newacronym{gr}{GR}{Kanton Graubünden}
% \newacronym{nw}{NW}{Kanton Nidwalden}
% \newacronym{ow}{OW}{Kanton Obwalden}
% \newacronym{sz}{SZ}{Kanton Schwyz}
% \newacronym{ur}{UR}{Kanton Uri}
% \newacronym{vs}{VS}{Kanton Wallis}
\begin{document}


\pagenumbering{gobble} % Suppress page numbering

\maketitle
% \addcontentsline{toc}{section}{Inhaltsverzeichnis}
\tableofcontents

% \include{sections/glossar}

\begin{abstract}
  
  \addsec{Abstract}
  % \addcontentsline{toc}{section}{Abstract}
  In vorliegender Arbeit wird die methodische \footnote{Wie können Risikowerte berechnet werden} sowie technische Umsetzung \footnote{Wie kann ein Computer diese Berechnungen ausführen} eines Computer-Programms zur automatisierten Planung von Ski- und Bergtouren realisiert. 
  \\
  Auf Basis des digitalen Höhenmodels (DEM) SwissAlti\textsuperscript{3D} mit einer Auflösung von 0.5m, der Bodenbedeckungskarte SwissTLM\textsuperscript{3D} sowie historischen Unfalldaten des SLF wird eine computeroptimierte Reduktionsmethode entwickelt, welche flächendeckend individuelle Gefahrenwerte für einzelne Rasterpunkte innerhalb der Schweiz errechnen kann. Um sinnvolle Risikowerte zu errechnen, muss ausserdem die Begehungshäufigkeit zugezogen werden. (Weniger Begehungen entsprechen nicht unbedingt linear auch weniger Unfällen). Die Berechnung der Risikokarten soll dabei in Echtzeit erfolgen können, um das täglich erscheinende Lawinenbulletin sowie – als Erweiterung des Projekts – im Tagesverlauf wahrgenommene Warnzeichen in die Karte aufnehmen zu können.
  
  Aus obigen Datengrundlagen können mittels vom Benutzer eingetragenen Start- \& Zielkoordinaten sichere, bzw. risikooptimierte Routen automatisiert geplant werden.
  Ausserdem wird der Einfluss eines solchen Werkzeuges auf die Risikobereitschaft eines Tourengängers sowie dessen Nutzwert diskutiert. Die mit dem Algorithmus erstellten Routen sollen durch Bergführer, Risk-V ausgebildete Schneesportlehrpersonen sowie Freizeittourengeher blind bewertet und aus einer Auswahl von nicht-computergenerierten Routen identifiziert werden.
\end{abstract}
	
\addsec{Vorwort}
Jedes Jahr ereignen sich in den Alpen abseits der gesicherten Pisten der Skigebiete XXX tödliche verlaufende Unfälle. Fortschritte 
\pagenumbering{arabic}
\section{Einleitung}
\subsection{Theoretische Grundlagen}
\subsubsection{Analytische vs. Praktische Lawinenkunde}
Lawinenprobleme lassen sich grundsätzlich aus zwei Perspektiven betrachten und vorhersagen: Einerseits auf Basis der Schneedecke, anderer seits auf Basis von Geländeformen und Statistik.
Dabei wird bei der praktischen Lawinenkunde auf das erstellen von beispielsweise Schneeprofilen (Extended Coloumn Test) zur Einschätzung der lokalen Stabilität der Schneedecke gesetzt.

Bei der analytischen Lawinenkunde arbeiten wir mit historischen Unfalldaten, konkret ist das Ziel, ein qunatitatives Mass für das eingegangene Riskio zu errechnen.
Wir korrelieren also bekannte konsequenzen mit der Hangform und deren Eintrittswahrscheinlichkeit~\cite{iso_risk}. Es folgt also unser Risikobegriff:
\begin{equation}
  r = k \cdot \frac{u}{b}
\end{equation}
wobei $r$ unser Risiko darstellt, $u$ die Anzahl Unfälle und $b$ die Anzahl Begehungen.
Anders ausgedrückt, misst das Risiko nicht in welchem Anteil der Begehungen ein Unfall eintritt, sondern, wie gross ein eintretender Unfall statistisch sein wird.

Dies steht der praktischen Lawinenkunde gegenüber. Hier beobachten wir die Schneedecke und leiten daraus die Entscheidung ab, weiter aufzusteigen oder umzukehren – diese Entscheidung steht in beiden Gebieten im Vordergrund.
\columnbreak{}

\subsubsection{Lawinentypen}

\begin{enumerate}
  \item Schneebrettlawinen~\cite{sacbergspwinter}\cite{slfLawinentypen}:
  \begin{itemize}
    \item Fordern 90\% der Lawinenopfer
    \item Abriss entlang einer Kante, Schnee gleitet als ganzer Block <<Brett>> ab
    \item Brett gleitet auf einer darunterliegenden Schwachschicht in der Schneedecke ab
    \item Kann aufgrund von mehrbelastung durch Wintersportler oder spontan abgehen
    \item Auslösender Athlet steht oft mitten im Schneebrett
    \item Verschüttungsgefahr gross –\\ Mitreis- \& Absturzgefahr gross
    \item Gefahr ab einer Hangneigung von $30\degree$
  \end{itemize}
    
  \item Lockerschneelawinen~\cite{sacbergspwinter}\cite{slfLawinentypen}:
  \begin{itemize}
    \item Punktförmiger Auslösepunkt
    \item Reisst immer mehr Schnee mit, Kegelförmiger Abgang der nach unten breiter wird
    \item Verschüttungsgefahr klein –\\ Mitreis- \& Absturzgefahr gross
    \item Im Auslösepunkt ist meistens eine hohe Steigung von $40\degree$ notwendig
  \end{itemize}

  \item Gleitschneelawinen~\cite{sacbergspwinter}\cite{slfLawinentypen}:
  \begin{itemize}
    \item Ebenfalls linienförmige Abrisskante
    \item Die gesamte Schneedecke gleitet ab
    \item Kleine Bedeutung für Wintersportler – gehen oft spontan ab und gefährden vor allem Infrastruktur
  \end{itemize}
\end{enumerate}
Für unsere Zwecke interessieren uns an erster Stelle \textbf{Schneebrettlawinen} und an zweiter \textbf{Lockerschneelawinen}.

\pagebreak
\subsubsection{Typische Lawinenprobleme}\label{lawinenprobleme}

\begin{enumerate}
  \item Neuschneeproblem~\cite{achtunglawine}:
  \begin{itemize}
    \item Frischer Niederschlag, der sich schlecht mit der darunterliegenden Schneeschichten verbunden hat. 
    \item Ab der Kritischen Neuschneemenge (je nach Bedingungen zwischen \qty{10}{cm} -- \qty{50}{cm}) mindestens die Warnstufe <<3 - Erheblich>> im LLB.
    \item Setzt sich in 2 -- 3 Tagen
  \end{itemize}
  \item Triebschneeproblem~\cite{achtunglawine}:
  \begin{itemize}
    \item Wind trägt Neuschnee in Windschattenlagen
    \item Praktisch immer gebunden, ideale Voraussetzungen für ein Schneebrett
    \item Gefahr durch Schneeverfrachtungen
    \item Setzt sich in 2 -- 3 Tagen
  \end{itemize}
  \item Nassschneeproblem~\cite{achtunglawine}:
  \begin{itemize}
    \item Die Schneedecke wird durch eindringendes Wasser geschwächt
    \item Erste Durchfeuchtung führt zu der bedeutendsten Schwächung
    \item Gefahr nach Regenperioden im Winter
    \item Typische Frühlingslawinen, Gefahr steigt im Verlauf des Tages an
    \item Lange anhaltende Gefahr
  \end{itemize}
  \item Altschneeproblem~\cite{achtunglawine}:
  \begin{itemize}
    \item z.B. Eingeschneite Harschschichten schwächen die Schneedecke erheblich
    \item Dauert Wochen -- Monate an
  \end{itemize}
\end{enumerate}
Es ist möglich, das mehrere Lawinenproblem miteinander auftreten. Im LLB sind jeweils alle zu erwartenden Lawinenprobleme mit einer eigenen Gefahrenstufe aufgeführt.~\cite{slfTypischeLawinenprobleme}

\subsubsection{Lawinenlagebericht}

Im Lawinenlagebericht, kurz LLB, werden die momentan am Berg zu erwartenden Lawinenprobleme (siehe~\ref{lawinenprobleme}) zusammengefasst und mit diversen zusätzlichen Daten ergänzt.
In der europäischen Lawinengefahrenskala werden fünf Warnstufen unterschieden, von <<1 - gering>> bis <<5 - sehr gross>>. \cite{lawinengefskala}

Rund 50\% der Unfälle geschehen dabei bei <<3 - mässig>>~\cite{achtunglawine}.

Die Skala kontinuierlich, jedoch nicht linear. Heisst konkret: 
Im Durchschnitt ist das Risiko bei <<mässig>> fünf Mal so hoch wie bei <<gering>>, und bei <<erheblich>> drei Mal so hoch wie bei <<mässig>>~\cite{sacbergspwinter}.

\begin{Figure}
  \centering
  \includegraphics[width=7.5cm]{gefahrenstufen}
  \captionof{figure}{Die 5 Europäsischen Lawinengefahrstufen in einem Graph}
\end{Figure}

Um die Einteilung dieser fünf Kategorien weiter zu verfeinern, wurden in der Schweiz zusätzlich Teilabstufen mittels $+$, $-$ und $=$ eingeführt (So lautet die Warnstufe im Bericht nun z.B.\ $3+$, wenn die tatsächliche Gefahr bereits nahe an einer $4$ bzw.\ $4-$ liegt)~\cite{sacbergspwinter}.
unterschiedliche Interpretation der standartisierten europäischen Skala sorgen leider dafür, das Differenzen zwischen Schweizerischen und Ausländischen Vorhersagen bestehen. Dies stellt besonders in Grenzregionen ein Problem dar, in dieser Arbeit im weiteren jedoch keine Beachtung geschenkt.


\subsubsection{Lawinenbildende Faktoren}

Lawinen sind das Resultat eines unglücklichen Zusammenspiels drei wesentlicher Faktoren: 
\textbf{Verhältnisse} (Wetterlage und Zustand der Schneedecke), \textbf{Gelände} (Neigung, Exposition, Höhe) und \textbf{Mensch} (95\% aller Lawinen mit Personenschaden wurden durch Menschen ausgelöst)~\cite{ortovoxlabsnow}.

Weder die Verhältnissen noch das Gelände lässt sich direkt von Wintersportlern beeinflussen. Nur durch die Wahl eines anderen Datums (nur in Ausnahmefällen sinnvoll, z.B.\ bei einer Erstbesteigung), oder durch clever Spuranlage lasen sich diese Faktoren mildern. 
Ist beides keine Option, muss eine alternative Tour ausgewählt werden.

\subsubsection{Schema 3x3}
Der Goldstandard der Tourenplanung ist heute Werner Munters $3\times3$-Schema. Dabei werden in drei <<Zoom>>-Stufen drei Faktoren ausgewertet:
\vfill
\textbf{Regional} 
Von zuhause aus~\cite{munter}:
\begin{itemize}
  \item Verhältnisse: 1. Lawinenlagebericht LLB, 2. Wetterprognose, 3. Auskünfte von Einheimischen/Hüttenwart
  \item Gelände: 1. $1:25000$-Karte, 2. Tourenführer, 3. Eigene Geländekentnisse
  \item Menschen: 1. Wer ist Dabei?, 2. Ausbildung, 3. Material, 4. Mentale und Phyische Kondition? 
\end{itemize}

\textbf{Lokal} Im Gebiet, auf Sichtdistanz~\cite{munter}\cite{redbull3x3}:
\begin{itemize}
  \item Verhältnisse / Schneedecke / Wetter: 1. Sicht?, 2. Bewölkung, Wind, Niederschlag, Temperatur?, 3. Schneeverfrachtungen, Neuschneemenge, 4. Stimmt der LLB?
  \item Gelände: 1. Stimmt meine Vorstellung (Steilheit, Exposition)? 2. Spuren anderer Gruppen
  \item Menschen: 1. Ausrüstungskontrolle (Gruppencheck LVS), 2. Andere Gruppen unterwegs?
\end{itemize}
\textbf{Zonal} Im Einzelhang~\cite{munter}\cite{redbull3x3}:
\begin{itemize}
  \item Verhältnisse: 1. Neuschneemenge, 2. Triebschnee, 3. Mögliche Abrisszonen, 4. Sonneneinstrahlung
  \item Gelände: 1. Wer / was ist über/unter der Gruppe?, 2. Steilste Stelle?, 3. Exposition, 4. Typisches Lawinengelände, 5. Hangform, 6. Höhe, 7. Oft befahren?
  \item Menschen: 1. Können \& Kondition, 2. Vorischtsmassnahmen, 3. Sichere Sammelstellen
\end{itemize}

\subsubsection{Digitale Höhenmodelle}

In den 1960er Jahren verlangte der schweizer Generalstab vom heutigen Bundesamt für Landestopografie zu prüfen, ob tief fliegende feindliche Kampfflugzeuge unbemerkt in die Schweiz eindringen konnten. Um die für diese Prüfung nötigen Berechnungen erstmals an einem Grossrechner ausführen zu können, mussten die topographischen Höhen aus der Landeskarte $1:250000$ auf Lochkarten transferiert werden. Mit einer Auflöung von \qty{250}{m} wurden die Höhenlinien der analogen Kartenprodukte so in mühseliger Handarbeit zwischen 1966 -- 1968 erstmals digital nutzbar gemacht. Das so erstellte digitale Höhenmodel DEM <<RIMINI>> wurde bis in die 1970er Jahre genutzt.~\cite{swisstopohistdem}

Rufe nach einem engmaschigerem Modell brachten schliesslich unter anderem <<DHM25>> hervor, welches eine Auflösung von bereits \qty{25}{m} mitbringt.~\cite{swisstopohistdem}

Um die Jahrtausendwende erfolgte dank schnelleren CPUs und günstigem Datenspeicher eine Umkehrung des Prozess. Neu werden die analogen Landeskarten auf Basis eines DEM erstellt. Moderne DEMs erreichen dabei eine Auflösung bis zu \qty{0.5}{m} bei einer Genauigkeit von 0.5 -- 3 \unit{m}~\cite{alti3dprod}. SwissAlti\textsuperscript{3D} ist genau solch ein Modell.

Dank der <<Open Government Data Strategie>> des Bundes werden seit 2020 diverse Datensammlungen die öffentliche Verwaltungen produzieren der Öffentlichkeit zur Verfügung gestellt~\cite{opendataswiss}.
So landet nebst dem Fahrplan der SBB, Jungwaldflächen und den Standorten aller öffentlichen Toiletten der Stadt Luzern auch  SwissAlti\textsuperscript{3D} auf dem Opendataprotal des Bundes.

SwissAlti\textsuperscript{3D} wird als GeoTIFF ausgeliefert. GeoTIFF sind letztendlich nur Bilddaten, die um einen Eintrag zur Lokalisierung in einem Koordinatensystem, hier LV95 LN02, ergänzt wurden. Der Farbwert eines Pixel entspricht jedoch der Höhe an dieser Stelle. Insgesamt wird die Schweiz in ca. $43500$ $\qty{1}{km} \times \qty{1}{km}$ grosse Kacheln unterteilt.~\cite{alti3dprod}

\subsubsection{Relevante Geländefaktoren}

\subsubsection{Unfalldaten \& Begehungen}


\subsection{Methodik}
\subsubsection{Berechnung von topographischen Oberflächenfaktoren}

Wir schneiden $3 \times 3$-Ausschnitt aus dem Höhenmodell. Unser Ziel ist es, die chrakteristischen Geländeeigenschaften für die Zelle $e$ zu berechen.
$a$ -- $h$ sind die Höhen der Gitterpunkte rund um unserer Zielzelle \textbf{e}:

\begin{Figure}
  \centering
  \includegraphics[width=7.5cm]{isoGrid}
  \captionof{figure}{$3 \times 3$-Ausschnitt von Höhen aus einem DEM und deren Benennung}
\end{Figure}

Hangneigung und Exposition nach \cite{gisslopeaspect}:

\begin{equation} \label{eq1}
  \frac{\Delta z}{\Delta x} = \frac{(c + 2f + i) - (a + 2d + g)}{8r}
\end{equation}
\begin{equation} \label{eq2}
  \frac{\Delta z}{\Delta y} = \frac{(g + 2h + i) - (a + 2b + c)}{8r}
\end{equation}

Hangneigung $\rho$ und Exposition $\theta$:
\begin{align}
  \rho &= \arctan \left( \sqrt{
    {\left( \frac{\Delta z}{\Delta x}\right)}^2 + 
    {\left(\frac{\Delta z}{\Delta y}\right)}^2}
  \right)\\
  \theta &= \arctan\left(\frac{\frac{\Delta z}{\Delta x}}{-\frac{\Delta z}{\Delta y}}\right)
\end{align}

Geländekrümmung nach~\cite{gismath}:
\begin{align}
  D &= \frac{{(d + f) / 2 - e}}{{r^2}} \\
  E &= \frac{{(b + h) / 2 - e}}{{r^2}} \\
  F &= \frac{{-a + c + g - i}}{{4r^2}} \\
  G &= \frac{{-d + f}}{{2r}} \\
  H &= \frac{{b - h}}{{2r}}
\end{align}

(6) -- (10) sind die Faktoren eines teilweisen Polynom vierten Grades~\cite{gismath}.
Hangkrümmung $c_{Plan}$ und $c_{Profil}$ beschreiben, mit welchem Radius sich die Hangneigung parallel (Plankrümmung) bzw.\ senkrecht (Profilkrümmung) zur Exposition ändert:

\begin{align}
    c_{Plan} &= -\frac{{2(DH^2 + EG^2 - FGH)}}{{G^2 + H^2}}
    \\
    c_{Profil} &= \frac{{2(DG^2 + EH^2 + FGH)}}{{G^2 + H^2}}
\end{align}

\subsubsection{Gefahrenwert als Funktion einer Gitterkoordinate}
Während der Realisierung unseres Programms stellten sich uns zwei Grundlegend verschiedene Herangehensweisen an die Bestimmung des Gefahrenwertes einer Gitterzelle. 

Die erste Variante versucht, das Gelände analog eines Menschen während einer Begehung zu beobachten und aufgrund von einigen wenigen markanten Werten eine Annahme zu treffen.

Die zweite Variante simuliert an jedem Punkt eine vereinfachten Lawinenabgang. Dabei werden die verschieden Parameter so justiert, das die simulierten Lawinenkegel in etwa echten Lawinenkegel entsprechen. Vereinfacht heisst in diesem Fall, dass sich eine Lawine wie eine Art <<rollender Ball>> verhält. Stellen, die von viele Bälle mit hoher durchschnittlicher Geschwischwindigkeit passiert werden, sind lawinengefärdeter als solche, welche nur mit geringer Geschwindigkeit und seltener passiert werden.

\subsubsection{Parallele Berechnungen}
\subsubsection{Risikoberechnung}
\subsubsection{3D-Karte im Webbrowser}
\section{Hauptteil}
\subsection{Resultat}

\subsection{Produktevaluation}

\subsection{(Methodische Reflexion)}
\section{Fazit \& Schlusswort}

Das in dieser Arbeit präsentierte Werkzeug kann nicht ohne weiteres bzw.\ nur mit entsprechender alpinistischer Ausbildung verwendet werden. Um zuverlässig begehbare Routen zu produzieren, ist ein komplexeres System, welches das Gelände anders als nur mit GRM und Anstrengungswerten <<begutachtet>> notwendig. Nebst Lawinengefahr ist auch die Absturzgefahr ein nicht zu unterschätzendes Kriterium. Das Werkzeug zeigt jedoch Potential, schon dieses zugegebenermassen ungeschickte Modell kann in einigen Gebieten auftrumpfen. Das einstellen der Gewichtungsparameter nahm einige Zeit in anspruch und es stellte sich als Schwierig heraus, eine gute Balance zwischen Risikominimierender Routenwahl sowie Wegdistanz zu finden. Der gesunde Menschenverstand, mag er manchmal auch emotional sein, behält zumindest momentan beim Tourenplanen noch die Oberhand.

\skiplines{3}
\large{
  \textit{<<Zu Risiken und Nebenwirkungen fragen Sie Ihren Berg- oder Tourenführer>>}
}
\clearpage


% \addcontentsline{toc}{section}{Literatur}
\printbibliography[heading=bibnumbered]
\clearpage

% \addcontentsline{toc}{section}{Abbildungsverzeichnis}
\appendix
\pagenumbering{Roman}
\glsaddallunused
\printglossary[type=\acronymtype, style=long, title={Abkürzungsverzeichnis}, nonumberlist]
\clearpage
\listoffigures

\skiplines{1}
Das Titelbild basiert auf einem mit Dall-E von OpenAI erstellten Rohbild. Promt <<Create me a romantic-style picture of a skitourer from behind, looking out into the mountains. The image should symbolize Algorithmic Planning of skitours -- from screen to the mountain and back safely>> Das Rohbild wurde manuell angepasst.\\

\skiplines{2}
Alle Abbildungen ohne Kurzverweis in der Beschriftung sind selbst erstellt. Die Hintergrundkarten stammen, wo nicht anders erwähnt, aus der <<LK-Winter>> von swisstopo. Die SAC-Routen sind den Tourenführern \citetitle{mmzentralch} von \citeauthor{mmzentralch}~\cite{mmzentralch}, \citetitle{classiquesrandoski} von \citeauthor{classiquesrandoski}~\cite{classiquesrandoski} sowie \citetitle{twslstgallappzll} von \citeauthor{twslstgallappzll}~\cite{twslstgallappzll} entnommen.
\clearpage

\section{Codebeispiele}
\subsection{Risikokarte}
\begin{figure}[H]
  \centering
  \includegraphics[width=0.9\linewidth]{code}
  \caption{Python Code zur Produktion von Risikokarten aus Neigungskarte}\label{fig:python}
\end{figure}

\section{Evaluationsrouten}\label{app:evalroutes}

\includepdf[landscape=true]{./../evaluation/PDFs/Clariden.pdf}
\includepdf[landscape=true]{./../evaluation/PDFs/Furka - Galenstock & Dammastock.pdf}
\includepdf[landscape=true]{./../evaluation/PDFs/Länta - Rheinwaldhorn_Adula.pdf}
\includepdf[landscape=true]{./../evaluation/PDFs/Monte Rosa Massiv.pdf}
\includepdf[landscape=true]{./../evaluation/PDFs/Piz Palü.pdf}
\includepdf[landscape=true]{./../evaluation/PDFs/Strahlhorn.pdf}
\includepdf[landscape=true]{./../evaluation/PDFs/Tierbergli.pdf}

\section{Lawinensimulation {RAMMS::Avalanche}}

\begin{figure}[H]
  \centering
  % \includegraphics[width=0.9\linewidth]{chelenalp_2d.gif}
  \caption{Python Code zur Produktion von Risikokarten aus Neigungskarte}\label{fig:chelenalp2d}
\end{figure}

\section{It's not a bug, it's a feature}

\end{document}
