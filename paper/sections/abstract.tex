\begin{abstract}
  
  \addsec{Abstract}
  % \addcontentsline{toc}{section}{Abstract}
  In der vorliegenden Arbeit wird Umsetzung eines Computerprogramms zur automatisierten Planung von Ski- und Bergtouren erörtert.
  Zwei Aspekte stehen dabei im Vordergrund: Methodik: Wie können Risikowerte berechnet werden? Technik: Wie kann ein Computer diese Berechnungen effizient ausführen?
  \par
  Auf Basis des digitalen Höhenmodells (\acrshort{dem} / \acrshort{dhm}) SwissAlti\textsuperscript{3D} mit einer Auflösung von 0.5m, der Bodenbedeckungskarte SwissTLM\textsuperscript{3D} sowie von historischen Unfalldaten des \gls{slf} wird eine computeroptimierte Reduktionsmethode entwickelt, welche flächendeckend individuelle Gefahrenwerte für einzelne Rasterpunkte innerhalb der Schweiz errechnen kann. Um sinnvolle Risikowerte zu errechnen, wird das Risiko aus historischen Unfalldaten und Begehungshäufigkeiten abgeleitet. Die Berechnung der Risikokarten soll dabei in Echtzeit erfolgen können, um das täglich erscheinende Lawinenbulletin sowie –- als Erweiterung des Projekts –- im Tagesverlauf wahrgenommene Warnzeichen in die Beurteilung aufzunehmen.
  
  Aus den obigen Datengrundlagen können mittels vom Benutzer eingetragenen Start- und Zielkoordinaten sichere, bzw.\ risikooptimierte Routen automatisiert geplant werden.
  Die resultierenden Routen werden mit Literaturtouren aus SAC-legitimierten Tourenführern verglichen und gegenüber diesen bewertet.
\end{abstract}