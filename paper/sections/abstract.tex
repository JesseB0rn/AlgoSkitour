\begin{abstract}
  
  \addsec{Abstract}
  % \addcontentsline{toc}{section}{Abstract}
  In vorliegender Arbeit wird die methodische\footnote{Wie können Risikowerte berechnet werden} sowie technische Umsetzung\footnote{Wie kann ein Computer diese Berechnungen effizient ausführen} eines Computer-Programms zur automatisierten Planung von Ski- und Bergtouren diskutiert. 
  \\
  Auf Basis des \acrlong{dhm} (\acrshort{dem} / \acrshort{dhm}) SwissAlti\textsuperscript{3D} mit einer Auflösung von 0.5m, der Bodenbedeckungskarte SwissTLM\textsuperscript{3D} sowie historischen Unfalldaten des SLF wird eine computeroptimierte Reduktionsmethode entwickelt, welche flächendeckend individuelle Gefahrenwerte für einzelne Rasterpunkte innerhalb der Schweiz errechnen kann. Um sinnvolle Risikowerte zu errechnen, wird das Risiko aus historischen Unfalldaten abgeleitet. Die Berechnung der Risikokarten soll dabei in Echtzeit erfolgen können, um das täglich erscheinende Lawinenbulletin sowie – als Erweiterung des Projekts – im Tagesverlauf wahrgenommene Warnzeichen in die Beurteilung aufnehmen zu können.
  
  Aus obigen Datengrundlagen können mittels vom Benutzer eingetragenen Start- \& Zielkoordinaten sichere, bzw.\ risikooptimierte Routen automatisiert geplant werden.
  Die resultierenden Routen werden mit Literaturtouren aus SAC-legitimierten Tourenführern verglichen und gegenüber diesen bewertet.
\end{abstract}