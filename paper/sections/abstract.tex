\begin{abstract}
  
  \addsec{Abstract}
  % \addcontentsline{toc}{section}{Abstract}
  In vorliegender Arbeit wird die methodische \footnote{Wie können Risikowerte berechnet werden} sowie technische Umsetzung \footnote{Wie kann ein Computer diese Berechnungen ausführen} eines Computer-Programms zur automatisierten Planung von Ski- und Bergtouren realisiert. 
  \\
  Auf Basis des digitalen Höhenmodels (DEM) SwissAlti\textsuperscript{3D} mit einer Auflösung von 0.5m, der Bodenbedeckungskarte SwissTLM\textsuperscript{3D} sowie historischen Unfalldaten des SLF wird eine computeroptimierte Reduktionsmethode entwickelt, welche flächendeckend individuelle Gefahrenwerte für einzelne Rasterpunkte innerhalb der Schweiz errechnen kann. Um sinnvolle Risikowerte zu errechnen, muss ausserdem die Begehungshäufigkeit zugezogen werden. (Weniger Begehungen entsprechen nicht unbedingt linear auch weniger Unfällen). Die Berechnung der Risikokarten soll dabei in Echtzeit erfolgen können, um das täglich erscheinende Lawinenbulletin sowie – als Erweiterung des Projekts – im Tagesverlauf wahrgenommene Warnzeichen in die Karte aufnehmen zu können.
  
  Aus obigen Datengrundlagen können mittels vom Benutzer eingetragenen Start- \& Zielkoordinaten sichere, bzw. risikooptimierte Routen automatisiert geplant werden.
  Ausserdem wird der Einfluss eines solchen Werkzeuges auf die Risikobereitschaft eines Tourengängers sowie dessen Nutzwert diskutiert. Die mit dem Algorithmus erstellten Routen sollen durch Bergführer, Risk-V ausgebildete Schneesportlehrpersonen sowie Freizeittourengeher blind bewertet und aus einer Auswahl von nicht-computergenerierten Routen identifiziert werden.
\end{abstract}
	