\section{Fazit \& Schlusswort}

Das in dieser Arbeit präsentierte Werkzeug kann nicht ohne Weiteres bzw.\ nur mit entsprechender alpinistischer Ausbildung verwendet werden. Um zuverlässig begehbare Routen zu produzieren, ist ein komplexeres System, welches das Gelände anders als nur mit \gls{grm} und Anstrengungswerten <<begutachtet>>, notwendig. Nebst Lawinengefahr ist auch die Absturzgefahr ein nicht zu unterschätzendes Kriterium. Das Werkzeug zeigt jedoch Potential: Bereits dieses zugegebenermaßen ungeschickte Modell kann in einigen Gebieten auftrumpfen. Das Einstellen der Gewichtungsparameter nahm einige Zeit in Anspruch und es stellte sich als schwierig heraus, eine gute Balance zwischen risikominimierender Routenwahl sowie Wegdistanz zu finden --- ein automatisches System gilt es zukünftig zu prüfen. Vorschläge für die weitere Verfolgung finden sich unter \ref{sec:improvements}. Die große Schwäche des Computers ist die fehlende Intuition; der gesunde Menschenverstand. Mag er manchmal auch emotional sein, behält er zumindest momentan beim Tourenplanen noch die Oberhand. 

Nebst den praktischen Einschränkungen kann auch über die philosophischen Hintergründe gestritten werden, angesichts des viralen Hypes rund um AI-Tools und Automatisierung --- wollen wir unsere Kunstformen und Selbstausdrücke an Maschinen übergeben? Jede Tour ist letztendlich das Verlangen eines Menschen nach Abenteuer, nach Entdeckung und eigenen Erfahrungen. Ich persönlich möchte, dass mir ein Computer die Arbeit abnimmt, damit ich mich anderen Dingen wie Skitouren, Sport oder Kunst widmen kann --- nicht andersherum.

\skiplines{2}
\large{
  \textit{<<Zu Risiken und Nebenwirkungen fragen Sie Ihren Berg- oder Tourenführer>>}
}