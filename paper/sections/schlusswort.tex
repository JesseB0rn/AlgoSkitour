\section{Fazit \& Schlusswort}

Das in dieser Arbeit präsentierte Werkzeug kann nicht ohne weiteres bzw.\ nur mit entsprechender alpinistischer Ausbildung verwendet werden. Um zuverlässig begehbare Routen zu produzieren, ist ein komplexeres System, welches das Gelände anders als nur mit GRM und Anstrengungswerten <<begutachtet>> notwendig. Nebst Lawinengefahr ist auch die Absturzgefahr ein nicht zu unterschätzendes Kriterium. Das Werkzeug zeigt jedoch Potential, schon dieses zugegebenermassen ungeschickte Modell kann in einigen Gebieten auftrumpfen. Das einstellen der Gewichtungsparameter nahm einige Zeit in anspruch und es stellte sich als Schwierig heraus, eine gute Balance zwischen Risikominimierender Routenwahl sowie Wegdistanz zu finden. Der gesunde Menschenverstand, mag er manchmal auch emotional sein, behält zumindest momentan beim Tourenplanen noch die Oberhand.

\skiplines{3}
\large{
  \textit{<<Zu Risiken und Nebenwirkungen fragen Sie Ihren Berg- oder Tourenführer>>}
}