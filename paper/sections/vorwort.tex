% \begin{multicols}{2}

\addsec{Vorwort}


Seit einigen Jahren, nicht zuletzt aufgrund der weltweiten Coronapandemie, erlebt das Skitourengehen eine neue Hochkonjunktur. Viele Pistenskifahrer stiegen wegen geschlossener Transportanlagen auf Muskelkraft um~\cite{sacCoronaTrend}. So begab auch ich mich mit meinem Vater auf meine erste Skitour – den 1922 Meter hohen Fronalpstock im Herzen des Kantons Schwyz. Eigentlich ist dieser Berg durch eine Sesselbahn in zwei Sektionen mechanisch erschlossen, nicht aber während der Weihnachtsferien 2020.
Als ich im August 2021 an der Alten Kanti Aarau mit dem Gymnasium begann und ich mich bereits in den ersten Wochen für eine Themenwoche entscheiden musste, schafften es für mich nur drei Angebote in die engere Auswahl: Die beiden Skilager Flims 1 \& 2 – und das Tourenlager Steingletscher, angeboten von Michael Kappeler. So war ich dann im April 2021 zum ersten Mal in Begleitung eines Bergführers auf Tour im Sustengebiet. Seither lässt mich das Skitourengehen nicht mehr los.

Im Winter der Jahre 2023/24\footnote{1. Oktober 2023 -- 26. März 2024} wurden vom WSL-Institut für Schnee- und Lawinenforschung \gls{slf} 175 Lawinenabgänge mit Sach- oder Personenschaden gemeldet. Insgesamt waren 190 Personen von einer Lawine erfasst worden.~\cite{slfWinterbericht202324}

Daher kommt auch die Motivation für diese Arbeit. Um eine risikominimierende Route zu planen, gibt es bereits viele sogenannte Reduktionsmethoden und Entscheidungshilfen, die dazu dienen, eine Ja/Nein-Entscheidung zu treffen, ob eine gewünschte Route möglich oder unmöglich ist~\cite{skitourenguruReduktionsmethoden}\cite{harveyrhynerschweizerlawinenkunde}.
Hier besteht oft eine gewisse Interpretationsfreiheit. Computer besitzen keine Arroganz, sie sind emotional nicht an eine Lieblingstour oder ein Gipfelerlebnis gebunden. Sie führen nur genau diese Berechnungen durch, mit denen sie beauftragt wurden – genau diese Qualität ist hier verlangt: stures, objektives Abarbeiten des Geländes auf der Suche nach Schlüsselstellen oder übermässig gewagten Routenführungen~\cite{harveyrhynerschweizerlawinenkunde}. Kann ein Computer also, ohne je das Konzept eines Gipfels zu verstehen, eine Reduktionsmethode durchführen und eine sinnvolle Route planen? Uns vom Bildschirm lösen, uns aus dem Wohnzimmer auf eine sichere Tour schicken? 
Auf der Suche nach einer variantenreichen Abfahrt das Risiko minimieren, in einer Statistik des \gls{slf} zu enden?

Einen grossen Dank an Christophe Seiler für die zahlreichen germanistischen Inputs. Ebenfalls an Eliane Häfliger für die treue Begleitung auf Skitour, im Alltag und nicht zuletzt bei der Korrektur dieser Arbeit. Ein Dank auch an Michael Kappeler für die Betreuung dieser Arbeit und der Touren, die wir zusammen unternommen haben.
