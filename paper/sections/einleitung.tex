\section{Einleitung}
\subsection{Theoretische Grundlagen}
\subsubsection{Einführung in digitale Höhenmodelle \& GIS}
Seit 20xx ist in der Schweiz das digitale Höhenmodel DHM25 verfügbar. Mit einer Auflösung von m werden aus den Höhenlinien der $1 : 25000$-Karte der Schweiz die geodätischen Höhen Computern nutzbar gemacht.
Für viele Anwendungen reicht bereits diese Auflösung aus. Fortschritte in Stereokorrelation, LIDAR und gänzlich neue Vermessungsverfahren ermöglichen heute jedoch Auflösungen unter 0.5\unit{m} mit einer Standartabweichung $\sigma \leq 0.3\unit{m}$.

\subsection{Methodik}
\lipsum[6-10]
